The RAiD Metadata Schema is divided into 'Core', 'Extended', and 'Local' components. The full metadata schema is available at: \href{https://metadata.raid.org/}{https://metadata.raid.org/}, and controlled vocabularies have been published via Research Vocabularies Australia \href{https://vocabulary.raid.org}{https://vocabularies.raid.org}.

\textbf{Core components} are metadata properties and associated controlled lists that are standardised across all RAiD Registration Agencies. If a Registration Agency uses its own controlled lists, it must provide a crosswalk to the RAiD’s standard terms.

\textbf{Extended components} are properties that are standardised across all Registration Agencies, but associated controlled lists may vary. A Registration Agency may use its own controlled lists if it publishes each list in a machine-readable format and registers the controlled list(s) with the ARDC. 

\textbf{Local components} are properties and controlled lists that are entirely under the control of a Registration Agency, and only need to be reported to the ARDC. Local properties can be tailored to meet the needs of the research community served by the Registration Agency. A mechanism exists to promote Local metadata properties should they prove useful.

Metadata is further organised into ‘blocks’ that include related properties, such as controlled lists, bounding dates, and language. In general, a typical pattern includes an identifier for an item in a controlled list, plus a link to the schema that is the source of the controlled list. Every free-text field also has a language attribute.